%% LyX 2.2.3 created this file.  For more info, see http://www.lyx.org/.
%% Do not edit unless you really know what you are doing.
\documentclass[11pt,a4paper,english,american]{article}
\usepackage{lmodern}
\renewcommand{\sfdefault}{lmss}
\renewcommand{\ttdefault}{lmtt}
\usepackage[T1]{fontenc}
\usepackage[latin9]{inputenc}
\usepackage{amsmath}
\usepackage{amssymb}

\makeatletter

%%%%%%%%%%%%%%%%%%%%%%%%%%%%%% LyX specific LaTeX commands.
\pdfpageheight\paperheight
\pdfpagewidth\paperwidth


%%%%%%%%%%%%%%%%%%%%%%%%%%%%%% Textclass specific LaTeX commands.
\usepackage{jcappub}
\numberwithin{equation}{section}

%%%%%%%%%%%%%%%%%%%%%%%%%%%%%% User specified LaTeX commands.
\pdfoutput=1
\renewcommand\[{\begin{equation}}
\renewcommand\]{\end{equation}} 
\renewcommand*\arraystretch{1.5}

\makeatother

\usepackage{babel}
\begin{document}

\title{Unbraiding the Bounce }

\subheader{preprint number }

\author[a]{David A. Dobre,}

\author[a]{Andrei V. Frolov,}

\author[a,b]{Jos\'{e} T. G\'{a}lvez Ghersi,}

\author[c]{Sabir Ramazanov,}

\author[c]{and Alexander Vikman}

\affiliation[a]{Department of Physics, Simon Fraser University, }

\affiliation{8888 University Drive, Burnaby, British Columbia V5A 1S6, Canada\\
 }

\affiliation[b]{Perimeter Institute for Theoretical Physics,}

\affiliation{31 Caroline Street North, Waterloo, Ontario, N2L 2Y5, Canada\\
}

\affiliation[c]{CEICO-Central European Institute for Cosmology and Fundamental Physics, }

\affiliation{Institute of Physics, the Academy of Sciences of the Czech Republic,
\\
Na Slovance 2, 182 21 Prague 8, Czech Republic\\
}

\emailAdd{ddobre@sfu.ca}

\emailAdd{frolov@sfu.ca}

\emailAdd{joseg@sfu.ca}

\emailAdd{ramazanov@fzu.cz}

\emailAdd{vikman@fzu.cz}

\abstract{We study a recently proposed by Ijjas and Steinhardt particular realization
\cite{Ijjas:2016tpn} of the cosmological bounce scenario. First,
we reveal the exact construction of the Lagrangian used in \cite{Ijjas:2016tpn}.
This explicit construction allowed us to study other cosmological
solutions in this theory. In particular we found solutions with superluminal
speed of sound and discuss the consequences of this feature for a
possible UV-completion. Further, following the originally constructed
background history, we evaluated the tensor and scalar spectra during
the bouncing phase characterized by the violation of the null energy
condition. We found that the change of the speed of sound is the cause
of the dominance of the tensor power spectrum over the scalar part
through most of the bounce. Moreover, we observe that none of the
spectra evaluated across the bouncing phase is scale invariant. In
addition to this, we present our results for particle production by
showing the evolution of the occupation number of scalar fluctuations
through the bounce. }
\maketitle

\section{Introduction}

Since 2011 it is well known that scalar-tensor theories with Kinetic
Gravity Braiding (which are also called first two terms of Horndeski
or simple generalized Galileons) allow for \cite{Qiu:2011cy,Easson:2011zy}
a classical bouncing evolution manifestly free from ghost and gradient
instabilities around the bounce. The possibility of the bounce in
such systems was briefly mentioned in Moreover, in \cite{Easson:2011zy}
it was demonstrated that one can easily construct spatially flat bouncing
universes, without ghosts and gradient instabilities and with bouncing
solutions of a non-vanishing measure. Moreover, in \cite{Easson:2011zy}
it was demonstrated that there is a continuum of such minimally coupled
theories of the type of Kinetic Gravity Braiding. This Ref. \cite{Easson:2011zy}
provided inequalities on two free functions of kinetic term present
in Lagrangian of the theory, which are sufficient to guaranty a healthy
bounce. It was also showed that this setup also works in a presence
(in fact unavoidable) of the normal matter - like radiation etc. In
one of examples (\textquotedbl{}Hot G-Bounce\textquotedbl{}) it was
even showed that such a bouncing universes can smoothly transit to
the radiation dominated stage. In this work it was also discussed
that there is always a singularity (either in the gravitational or
acoustic metric). 

\section{Model and main equations }

In 2016, Ijjas and Steinhardt proposed a particular realization \cite{Ijjas:2016tpn}
of the cosmological bounce scenario in a particular subclass of these
theories. For convenience we denote this realization as \emph{IS-bounce}.
The IS-bounce was claimed to be free from ghost and gradient instabilities
and free from superluminal propagation of perturbations. The explicit
case of the IS-bounce uses a class of Kinetic Gravity Braiding theories
with explicitly strongly \emph{broken} shift-symmetry $\phi\rightarrow\phi+c$
\begin{equation}
S=\frac{1}{2}\int d^{4}x\,\sqrt{-g}\left(k\left(\phi\right)\left(\partial\phi\right)^{2}+\frac{1}{2}q\left(\phi\right)\left(\partial\phi\right)^{4}+\left(\partial\phi\right)^{2}\Box\phi\right)\,,\label{eq:IS_Lagrangian}
\end{equation}
where 
\[
\left(\partial\phi\right)^{2}\equiv g^{\mu\nu}\partial_{\mu}\phi\partial_{\nu}\phi\equiv2X\,,\qquad\Box\phi\equiv g^{\mu\nu}\nabla_{\mu}\nabla_{\nu}\phi\,,
\]
where $\nabla_{\mu}$ is the usual Levi-Civita connection \foreignlanguage{english}{}\footnote{Further we use: the standard notation $\sqrt{-g}\equiv\sqrt{-\text{det}g_{\mu\nu}}$
where $g_{\mu\nu}$ is the metric, the signature convention $\left(+,-,-,-\right)$
(contrary to \cite{Ijjas:2016tpn}), and the units $c=\hbar=1$, $M_{\text{Pl}}=\left(8\pi G_{\text{N}}\right)^{-1/2}=1$. }. The scalar field is supposed to be minimally coupled to gravity.
Hence the theory is defined by two free functions $k\left(\phi\right)$
and $q\left(\phi\right)$. In notation of \cite{Deffayet:2010qz}
we have\footnote{At the beginning the authors of \cite{Ijjas:2016tpn} also used $G\left(X,\phi\right)=b\left(\phi\right)X$,
however this additional free function $b\left(\phi\right)$ can be
eliminated by a simple field-redefinition. } 
\begin{equation}
K\left(X,\phi\right)=k\left(\phi\right)X+q\left(\phi\right)X^{2}\,,\qquad G\left(X,\phi\right)=X\,.\label{eq:KGB_identifiaction}
\end{equation}
This identification allows us to directly use all necessary formulas
derived in \cite{Deffayet:2010qz} for arbitrary $K\left(X,\phi\right)$
and $G\left(X,\phi\right)$, see also \cite{Kobayashi:2010cm}. In
particular the pressure is 
\[
P=kX+qX^{2}-2X\ddot{\phi}\text{�,}
\]
\[
J=\dot{\phi}\left(k+2qX+3\dot{\phi}H\right)\text{�,}
\]
the energy density 
\[
\varepsilon=\dot{\phi}J-kX-qX^{2}
\]
\[
\dot{H}=X\ddot{\phi}-\frac{1}{2}\dot{\phi}J\text{�,}
\]

The quadratic action for curvature perturbations $\mathcal{R}$ reads\footnote{Note that \cite{Ijjas:2016tpn} does not use the canonical normalization
with $1/2$ in front of the action. Hence our coefficients $A\left(t\right)$
and $B\left(t\right)$ are twice larger then those in \cite{Ijjas:2016tpn}. } 
\begin{equation}
S_{2}=\frac{1}{2}\int dt\,d^{3}x\,a^{3}\left(A\left(t\right)\dot{\mathcal{R}}^{2}-\frac{B\left(t\right)}{a^{2}}\left(\partial_{i}\mathcal{R}\right)^{2}\right)\,.\label{eq:quadratic_action_scalars}
\end{equation}
The formula for the normalization of the kinetic term is given by
(A.9) page 36, \cite{Deffayet:2010qz}
\begin{equation}
A=\frac{2XD}{\left(H-\dot{\phi}XG_{,X}\right)^{2}}\,,\label{eq:Our_Normalization}
\end{equation}
where the quantity 
\begin{equation}
D=K_{,X}+2XK_{,XX}-2G_{,\phi}-2XG_{,X\phi}+6\dot{\phi}H\left(G_{,X}+XG_{,XX}\right)+6X^{2}G_{,X}^{2}\,,\label{eq:D_for_ghosts}
\end{equation}
determines ($D>0$) whether the perturbations are free of ghosts.
The sound speed is given by the formula (A.11) page 36, \cite{Deffayet:2010qz}
\begin{equation}
c_{s}^{2}=\frac{B\left(t\right)}{A\left(t\right)}=\frac{\dot{\phi}XG_{,X}\left(H-\dot{\phi}XG_{,X}\right)-\partial_{t}\left(H-\dot{\phi}XG_{,X}\right)}{XD}\,,\label{eq:speed_of_sound}
\end{equation}
where we assumed that the field $\phi$ is the only source of energy-momentum.
From this expression we can write 
\begin{equation}
\frac{1}{2}B\left(t\right)=\frac{\dot{\phi}XG_{,X}\left(H-\dot{\phi}XG_{,X}\right)-\partial_{t}\left(H-\dot{\phi}XG_{,X}\right)}{\left(H-\dot{\phi}XG_{,X}\right)^{2}}\text{�.}\label{eq:B_kgb}
\end{equation}
Hence it is natural to introduce a convenient variable
\begin{equation}
\gamma=H-\dot{\phi}XG_{,X}\text{�,}\label{eq:gamma}
\end{equation}
in terms of which 
\begin{equation}
\frac{1}{2}B\left(t\right)=\frac{d}{dt}\gamma^{-1}+H\gamma^{-1}-1\text{�.}\label{eq:B_general}
\end{equation}
The expression for $B\left(t\right)$ was written in this elegant
form in \cite{Ijjas:2016tpn} for a particular choice (\ref{eq:KGB_identifiaction})
of functions $K$ and $G$. However, before that this variable (\ref{eq:gamma})
was used in \cite{Libanov:2016kfc}, see equation (2.6) page 4. 

For the Lagrangian given by (\ref{eq:KGB_identifiaction}) one obtains
\begin{equation}
\gamma\left(t\right)=H-\dot{\phi}X\,,\label{eq:Gamma_IS_Lagrangian}
\end{equation}
so that (\ref{eq:B_kgb}) (or (\ref{eq:B_general})) yields
\begin{equation}
B=\frac{2X\left(k+2qX+4H\dot{\phi}+2\ddot{\phi}-2X^{2}\right)}{\left(H-\dot{\phi}X\right)^{2}}\,.\label{eq:B_IS}
\end{equation}
While the general expression (\ref{eq:D_for_ghosts}) gives 
\begin{equation}
D=k+6Xq+6\dot{\phi}H+6X^{2}\,,\label{eq:D_IS}
\end{equation}
 and consequently (\ref{eq:Our_Normalization}) reads 
\begin{equation}
A=\frac{2X\left(k+6qX+6\dot{\phi}H+6X^{2}\right)}{\left(H-\dot{\phi}X\right)^{2}}\,.\label{eq:Normalization}
\end{equation}

\section{Inverse method, finding the theory }

The authors of IS-bounce pustulated a particular fairly simple time-dependence
of the Hubble parameter 
\begin{equation}
H\left(t\right)=H_{0}\,t\,\exp\left(-F\left(t-t_{*}\right)^{2}\right)\,,\label{eq:Hubble_IS}
\end{equation}
where $H_{0}$, $F$ and $t_{*}$ are constants, and proposed an ``inverse
method'' to find free functions $k\left(\phi\right)$ and $q\left(\phi\right)$
in (\ref{eq:IS_Lagrangian}) which can realize this cosmological evolution.
 

The key observation of the ``inverse method'' proposed in \cite{Ijjas:2016tpn}
is that one can can also independently postulate $\gamma\left(t\right)$
in (\ref{eq:B_general}). The IS-bounce postulates 
\begin{equation}
\gamma=\gamma_{0}\exp\left(3\theta t\right)+H\left(t\right)\,,\label{eq:Gamma _IS}
\end{equation}
where $\gamma_{0}$ and $\theta$ are additional constants with respect
to already introduced $H_{0}$, $F$ and $t_{*}$. From (\ref{eq:Gamma_IS_Lagrangian})
one can obtain 
\[
\phi\left(t\right)=\phi_{0}+\int_{t_{0}}^{t}dt'\left(2\left(H-\gamma\right)\right)^{1/3}\,,
\]
where $\phi\left(t_{0}\right)=\phi_{0}$. It is convenient to chose
this initial value as $\phi_{0}=\left(-2\gamma_{0}/\theta\right)^{1/3}\exp\left(3t_{0}\right),$
so that the particular solution postulated in IS-bounce is 
\begin{equation}
\phi\left(t\right)=\left(\frac{-2\gamma_{0}}{\theta^{3}}\right)^{1/3}\exp\left(\theta t\right)\,.\label{eq:phi_solution}
\end{equation}
For the later it is suitable to denote 
\[
\phi_{\star}=\left(\frac{-2\gamma_{0}}{\theta^{3}}\right)^{1/3}\,,
\]
as a characteristic field range, so that 
\begin{equation}
t=\frac{1}{\theta}\text{log}\left(\phi/\phi_{\star}\right)\,.\label{eq:time_IS_solution}
\end{equation}
Using the substitutions (\ref{eq:Hubble_IS}) and (\ref{eq:Gamma _IS})
one obtains the functions $k\left(\phi\right)$ and $q\left(\phi\right)$
as functions of time on the particular solution (\ref{eq:phi_solution})
\begin{equation}
k\left(t\right)=-\frac{2\left(2\dot{H}+3H^{2}+\dot{\gamma}+3H\gamma\right)}{\left(2\left(H-\gamma\right)\right)^{2/3}}\,,\label{eq:k(t)}
\end{equation}
and 
\begin{equation}
q\left(t\right)=\frac{4\left(2\dot{H}+\dot{\gamma}+9H\gamma\right)}{3\left(2\left(H-\gamma\right)\right)^{4/3}}\,.\label{eq:q(t)}
\end{equation}
For the later it is convenient to introduce functions 
\begin{equation}
W\left(\phi\right)=\exp\left[\frac{F}{\theta^{2}}\left(\log\left(\frac{\phi}{\phi_{\star}}\right)-\theta t_{*}\right){}^{2}\right]\,,\label{eq:W_IS}
\end{equation}
and 
\begin{equation}
\Omega\left(\phi\right)=W\left(\phi\right)\theta^{3}\left(\theta\phi\right)^{3}+H_{0}\left[\log\left(\frac{\phi}{\phi_{\star}}\right)\left(4F\left[\log\left(\frac{\phi}{\phi_{\star}}\right)-\theta t_{*}\right]+2\theta\left(\theta\phi\right)^{3}\right)-2\theta^{2}\right]\,,\label{eq:Omega_IS}
\end{equation}
in terms of which the defining functions are 
\begin{equation}
k(\phi)=-\frac{12H_{0}^{2}\log^{2}\left(\phi/\phi_{\star}\right)-3W\left(\phi\right)\left[\Omega\left(\phi\right)-H_{0}\theta\left(\theta\phi\right)^{3}\log\left(\phi/\phi_{\star}\right)\right]}{W^{2}\left(\phi\right)\theta^{2}\left(\theta\phi\right){}^{2}}\,,\label{eq:k(phi)}
\end{equation}
and 
\begin{equation}
q\left(\phi\right)=\frac{12H_{0}^{2}\log^{2}\left(\phi/\phi_{\star}\right)-2W\left(\phi\right)\left[\Omega\left(\phi\right)+H_{0}\theta\left(\theta\phi\right)^{3}\log\left(\phi/\phi_{\star}\right)\right]}{W^{2}\left(\phi\right)\theta^{2}\left(\theta\phi\right){}^{4}}\,.\label{eq:q(phi)}
\end{equation}
These expressions defining the theory which should be related to the
origins of the universe neither look well-motivated nor natural from
any point of view. This is the prize for the chosen simple exact solution
(\ref{eq:Hubble_IS}), (\ref{eq:phi_solution}). 

\section{Other Solutions, Phase Space }

\section{Superluminality}

It is known that generic theories with derivative interactions can
have such configurations that small perturbations propagate faster
than light. In k-essence one can formulate inequalities which can
guarantee the absence of such superluminality. However, this seems
to be impossible to achieve in general theories with Kinetic Gravity
Braiding. There are examples \cite{Creminelli:2012my,Easson:2013bda}of
such theories where there is no superluminality for \emph{all} cosmological
configurations, for the proof see\cite{Easson:2013bda}. However,
this only happens in an idealized universe without any external matter.
An unusual property of the theories with Kinetic Gravity Braiding
is that the value of the sound speed does depend not only on the local
state of the field $\left(\phi,\dot{\phi}\right)$, but also on the
external matter energy-momentum tensor. Even in the case of of the
subluminal cosmological states \cite{Creminelli:2012my} the external
matter introduces the superluminality at least for some regions of
phase space \cite{Easson:2013bda}. 

\section{Perturbations}

In this section, we describe the dynamics of the scalar and tensor
fluctuations by considering the IS-bouncing trajectory proposed in
\cite{Ijjas:2016tpn}, which was described in the previous sections.
All the information needed from this trajectory is contained in the
choice of $H(t)$ in (\ref{eq:Hubble_IS}) and $\gamma(t)$ in (\ref{eq:Gamma _IS}).
For instance, we will first describe the set of equations of motion
and initial conditions required for numerical evaluation. 

\subsection{Setting up the equations of motion and initial conditions}

The field dynamics of these fluctuations is described by the Lagrangian
in (\ref{eq:IS_Lagrangian}) expanded in second order of perturbations
written in Fourier space, 
\begin{equation}
\mathcal{L}_{\mathbf{k}}^{(2)}=\frac{a^{3}A(t)}{2}\left(\left|\dot{\mathcal{R}_{\mathbf{k}}}\right|^{2}-\frac{k^{2}c_{s}^{2}}{a^{2}}\left|\mathcal{R}_{\mathbf{k}}\right|^{2}\right)+\frac{a^{3}}{4}\sum_{p=+,\times}\left[\left|\dot{h}_{\mathbf{k}}^{p}\right|^{2}-\frac{k^{2}}{a^{2}}\left|h_{\mathbf{k}}^{p}\right|^{2}\right].\label{eq:IS_second_order}
\end{equation}
Where tensor modes are also included apart from the scalar curvature
shown in (\ref{eq:Scalar_curv}) and the index $p$ is a placeholder
for any of the polarization modes, conventionally dubbed by $(+)$
and $(\times)$. In this case, the speed of sound follows from its
previous definitions and $A(t)$ is the same as in (\ref{eq:Normalization}).
We derive the equations of motion after varying the action built from
the previous expression with respect to $\mathcal{R}_{\mathbf{k}}$
and $h_{\mathbf{k}}^{p}$, 
\begin{eqnarray*}
 & {\displaystyle {\ddot{\mathcal{R}}_{\mathbf{k}}+\frac{d\ln(a^{3}A(t))}{dt}\dot{\mathcal{R}}_{\mathbf{k}}+\frac{k^{2}c_{s}^{2}}{a^{2}}\mathcal{R}_{\mathbf{k}}=0,}}\\
 & {\displaystyle {\ddot{h}_{\mathbf{k}}^{p}+3H\dot{h}_{\mathbf{k}}^{p}+\frac{k^{2}}{a^{2}}h_{\mathbf{k}}^{p}=0.}}
\end{eqnarray*}
With the help of the auxiliary variables $\mathcal{R}_{\mathbf{k}}\equiv a^{-3/2}A(t)^{-1/2}\mathcal{S}_{\mathbf{k}}$
and $h_{\mathbf{k}}^{p}\equiv\sqrt{{2}}a^{-3/2}\mathcal{T}_{\mathbf{k}}^{p}$
it is possible to write these equations in the form of two decoupled
simple harmonic oscillators, 
\begin{eqnarray}
 & {\displaystyle {\ddot{\mathcal{S}}_{\mathbf{k}}+\omega_{\mathcal{S}}^{2}\mathcal{S}_{\mathbf{k}}=0,}\label{eq:IS_{e}qmov_{s}calar}}\\
 & {\displaystyle {\ddot{\mathcal{T}}_{\mathbf{k}}^{p}+\omega_{\mathcal{T}}^{2}\mathcal{T}_{\mathbf{k}}^{p}=0,}\label{eq:IS_{e}qmov_{t}ensor}}
\end{eqnarray}
where the functions $z\equiv a^{3/2}A^{1/2}$ and $y\equiv a^{3/2}$
are useful in order to define the natural frequencies for both oscillators
$\omega_{\mathcal{S}}^{2}\equiv k^{2}c_{s}^{2}/a^{2}-\ddot{z}/z$
and $\omega_{\mathcal{T}}^{2}\equiv k^{2}/a^{2}-\ddot{y}/y$. For
simplicity, we will evolve these auxiliary variables instead of the
original scalar and tensor modes; however it is not a problem to determine
the dynamics of $\mathcal{R}_{\mathbf{k}}$ and $h_{\mathbf{k}}^{p}$
after inverting the definitions of $\mathcal{S}_{\mathbf{k}}$ and
$\mathcal{T}_{\mathbf{k}}^{p}$. Initial conditions can be determined
by instantaneous energy minimization after diagonalizing the Hamiltonian
of an harmonic oscillator with a time-dependent frequency in a fixed
instant of time, even when it is well-known that this selection criteria
is not unique. The promotion of these solutions to field operators
is given by, 
\[
\hat{\mathcal{S}}_{\mathbf{k}}=\mathcal{S}_{\mathbf{k}}\hat{a}_{\mathbf{k}}+\mathcal{S}_{\mathbf{-k}}^{*}\hat{a}_{\mathbf{-k}}^{\dagger}~;~\hat{\mathcal{T}}_{\mathbf{k}}^{p}=\mathcal{T}_{\mathbf{k}}^{p}\hat{b}_{\mathbf{k}}+(\mathcal{T}_{\mathbf{k}}^{p})^{*}\hat{b}_{\mathbf{-k}}^{\dagger},
\]
and is consistent with the reality condition for all the fluctuation
modes. $(\hat{a}_{\mathbf{k}},\hat{a}_{\mathbf{k}}^{\dagger})$ and
$(\hat{b}_{\mathbf{k}},\hat{b}_{\mathbf{k}}^{\dagger})$ are the creation
and annihilation operators for scalar and tensor perturbations. Any
generic solution to (\ref{eq:IS_eqmov_scalar}) and (\ref{eq:IS_eqmov_tensor})
can be written as a function of two real modes $\mathcal{S}_{\mathbf{k}}=\mathcal{S}_{\mathbf{k}}^{(1)}+i\mathcal{S}_{\mathbf{k}}^{(2)}$.
From the standard commutation relations for the ladder operators $[\hat{a}_{\mathbf{k}},\hat{a}_{\mathbf{k}}^{\dagger}]=\delta(\mathbf{{k}-{k'}})$
and the equal-time commutator $[\hat{\mathcal{S}}_{\mathbf{k}},\dot{\hat{\mathcal{S}}}_{\mathbf{k'}}]=i\delta(\mathbf{{k}-{k'}})$,
we can show that the Wronskian of these real modes satisfies, 
\[
\mathcal{S}_{\mathbf{k}}^{(1)}\dot{\mathcal{S}}_{\mathbf{k}}^{(2)}-\mathcal{S}_{\mathbf{k}}^{(2)}\dot{\mathcal{S}}_{\mathbf{k}}^{(1)}=\frac{1}{2},
\]
at every instant of time, in agreement with the canonical commutations
relations. The same procedure follows to write conservation of the
Wronskian for the tensor modes. Hence, it is possible to write the
canonically normalized initial conditions for all the modes involved,
\begin{eqnarray}
 & {\displaystyle {\dot{\mathcal{S}}^{(1)}(t_{0})=0~;~\mathcal{S}^{(1)}(t_{0})=\frac{1}{\sqrt{2\omega_{\mathcal{S}}(t_{0})}}},\nonumber}\\
 & {\displaystyle {\dot{\mathcal{S}}^{(2)}(t_{0})=\sqrt{\frac{\omega_{\mathcal{S}}(t_{0})}{2}}~;~\mathcal{S}^{(2)}(t_{0})=0},\label{eq:IS_{i}nit_{S}}}\\
 & {\displaystyle {(\dot{\mathcal{T}}^{p})^{(1)}(t_{0})=0~;~(\mathcal{T}^{p})^{(1)}(t_{0})=\frac{1}{\sqrt{2\omega_{\mathcal{T}}(t_{0})}}},\nonumber}\\
 & {\displaystyle {(\dot{\mathcal{T}}^{p})^{(2)}(t_{0})=\sqrt{\frac{\omega_{\mathcal{T}}(t_{0})}{2}}}~;~(\mathcal{T}^{p})^{(2)}(t_{0})=0,\label{eq:IS_{i}nit_{T}}}
\end{eqnarray}
in an analogous way than the standard prescription for the Bunch-Davies
vacuum. Notice that even in the limit of high $k$, it is possible
to observe that $c_{s}^{2}(t)$ and $a(t)$ make the scalar initial
conditions not invariant under time translations. Our objective with
this choice of initial conditions is the instantaneous minimization
of the energy per $k$-mode for \emph{only} positive values of $\omega_{\mathcal{T}}^{2}$
and $\omega_{\mathcal{S}}^{2}$ before the bounce. In this way, we
will compare the state of the system at any posterior instant of time
with the zero-point energy state at $t=t_{0}$ of a time-dependent
harmonic oscillator. This would be a suitable (but not unique) way
to define equations of motion while setting initial conditions that
minimize the energy of the system before the bounce. Notice that the
evaluation of the expressions in (\ref{eq:IS_init_S}) and (\ref{eq:IS_init_T})
at all the posterior instants of time should never be treated as solutions
to the equations of motion. Nevertheless, there is another way to
rephrase the system that increases the performance of a numerical
computation. For instance, let us rewrite the eikonal separation $\mathcal{S}_{\mathbf{k}}=L_{\mathcal{S}}\exp(i\Theta_{\mathcal{S}})$,
where $L_{\mathcal{S}}$ and $\Theta_{\mathcal{S}}$ are real. Then
we replace this expression in (\ref{eq:IS_eqmov_scalar}) obtaining
two equations after separating the real and imaginary parts, 
\begin{eqnarray}
 & \ddot{L}_{\mathcal{S}}+(\omega_{\mathcal{S}}^{2}-\dot{\Theta}_{\mathcal{S}}^{2})L_{\mathcal{S}}=0~~\mathrm{(Real\:part)},\nonumber \\
 & \ddot{\Theta}_{\mathcal{S}}+2\frac{\dot{L}_{\mathcal{S}}}{L_{\mathcal{S}}}\dot{\Theta}_{\mathcal{S}}=0~~\mathrm{(Imaginary\:part)},\label{eq:IS_phase_S}
\end{eqnarray}
where (\ref{eq:IS_phase_S}) has a simple analytic solution given
by $\dot{\Theta}_{\mathcal{S}}(t)=L_{\mathcal{S}}^{2}(t_{0})\dot{\Theta}_{\mathcal{S}}(t_{0})/L_{\mathcal{S}}^{2}$.
Hence, the only equation to solve for the scalar modes is, 
\begin{equation}
\ddot{L}_{\mathcal{S}}+\left[\omega_{\mathcal{S}}^{2}-L_{\mathcal{S}}^{4}(t_{0})\frac{\dot{\Theta}_{\mathcal{S}}^{2}(t_{0})}{L_{\mathcal{S}}^{4}}\right]L_{\mathcal{S}}=0.\label{eq:IS_amp_S}
\end{equation}
The same idea can be straightforwardly applied for the tensor modes
after replacing all the tensor components of $\mathcal{T}_{\mathbf{k}}^{p}$
by $L_{\mathcal{T}}^{p}\exp(i\Theta_{\mathcal{T}})$. Assuming the
same contribution from all polarizations, this will lead us to an
analogous result for the phase velocity $\dot{\Theta}_{\mathcal{T}}(t)$
and the equation of motion for the amplitudes differ from (\ref{eq:IS_amp_S})
only by considering $\omega_{\mathcal{T}}^{2}$ instead of $\omega_{\mathcal{S}}^{2}$,
\begin{equation}
\ddot{L}_{\mathcal{T}}^{p}+\left[\omega_{\mathcal{T}}^{2}-(L_{\mathcal{T}}^{p})^{4}(t_{0})\frac{\dot{\Theta}_{\mathcal{T}}^{2}(t_{0})}{(L_{\mathcal{T}}^{p})^{4}}\right]L_{\mathcal{T}}^{p}=0.\label{eq:IS_amp_T}
\end{equation}
It is important to remember that the power spectrum only depends on
the amplitudes of the original modes, which can found at all times
by, 
\begin{equation}
|\mathcal{R}_{\mathbf{k}}|=\frac{|\mathcal{S}_{\mathbf{k}}|}{a^{3/2}A^{1/2}}=\frac{L_{\mathcal{S}}}{a^{3/2}A^{1/2}}\:,\:|h_{\mathbf{k}}^{p}|=\frac{\sqrt{{2}}|\mathcal{T}_{\mathbf{k}}^{p}|}{a^{3/2}}=\frac{\sqrt{{2}}L_{\mathcal{T}}^{p}}{a^{3/2}}.\label{eq:IS_amplitudes}
\end{equation}

Initial conditions in (\ref{eq:IS_init_S}) and (\ref{eq:IS_init_T})
are reproduced after evaluating $\Theta_{\mathcal{T}}(t_{0})=0$ and
$\Theta_{\mathcal{S}}(t_{0})=0$, similarily $\dot{L}_{\mathcal{T}}^{p}(t_{0})=0$
and $\dot{L}_{\mathcal{S}}(t_{0})=0$. The amplitudes of the modes
and the phase derivatives are the only initial conditions with non-trivial
values, 
\begin{eqnarray}
 & {L}_{\mathcal{S}}(t_{0})=\frac{1}{\sqrt{2\omega_{\mathcal{S}}(t_{0})}}~~;~~\dot{\Theta}_{\mathcal{S}}(t_{0})=\omega_{\mathcal{S}}(t_{0}),\label{eq:IS_real_ICs_scalar}\\
 & {L}_{\mathcal{T}}^{p}(t_{0})=\frac{1}{\sqrt{2\omega_{\mathcal{T}}(t_{0})}}~~;~~\dot{\Theta}_{\mathcal{T}}(t_{0})=\omega_{\mathcal{T}}(t_{0}),\label{eq:IS_real_ICs_tensor}
\end{eqnarray}
where the effective frequency of the harmonic oscillators in (\ref{eq:IS_amp_S})
and (\ref{eq:IS_amp_T}) at $t=t_{0}$ is exactly zero for the scalar
and tensor modes regardless of the values of $k$. This procedure
describes the single field realization of the separation technique
of fast and slow components shown in \cite{Ghersi:2016gee}. Our construction
relies heavily in the positivity of $\omega_{\mathcal{T}}^{2}$ and
$\omega_{\mathcal{S}}^{2}$ at $t=t_{0}$, for that reason the initial
instant of time where the mode evolution starts needs to be chosen
properly in order to avoid diverging initial conditions. Both functions
are positive in the range of very high values of $k$, therefore we
evaluated both effective frequencies in Figure \ref{fig:w_eff_ST}
in the low $k$ limit. Henceforth, we choose $t_{0}=-90\:t_{\mathrm{Pl}}$
to be the initial instant of time where the evolution of perturbations
begin. Moreover, it is also (approximately) the first moment where
both $A(t)$ and $c_{s}^{2}$ are finite and positive, thus the system
of scalar and tensor fluctuations evolves in a time domain where it
is free from ghost and gradient instabilities. 

\subsection{Evolution of the mode amplitudes and length scales}

The definitions of the effective oscillation frequencies for the scalar
curvature $(\omega_{\mathcal{S}}^{2})$ and primordial tensor $(\omega_{\mathcal{T}}^{2})$
modes shows that only the scalar fluctuations propagate with sound
speed $c_{s}$. It is relevant, therefore, to compare the evolution
of different physical length scales as these propagate from sub-horizon
scales. Let us define the two physical scales with respect to a given
comoving wavelength $\lambda_{\mathrm{com}}$ by 
\begin{eqnarray}
 & {\displaystyle {\lambda_{\mathrm{phys}}^{\mathcal{S}}=\frac{a\lambda_{\mathrm{com}}}{c_{s}},}\label{lambda_{S}}}\\
 & {\displaystyle {\lambda_{\mathrm{phys}}^{\mathcal{T}}=a\lambda_{\mathrm{com}},}\label{lambda_{T}}}
\end{eqnarray}
which can be identified as the corresponding wavelength part from
the definitions of $\omega_{\mathcal{S}}^{2}$ and $\omega_{\mathcal{T}}^{2}$.
We now evaluate the time derivative of the logarithmic versions of
these two quantities, obtaining 
\begin{eqnarray}
 & {\displaystyle {\frac{d\ln\lambda_{\mathrm{phys}}^{\mathcal{S}}}{dt}=H-\frac{1}{2}\left(\frac{\dot{B}}{B}-\frac{\dot{A}}{A}\right),}\label{lambda_{d}ot_{S}}}\\
 & {\displaystyle {\frac{d\ln\lambda_{\mathrm{phys}}^{\mathcal{T}}}{dt}=H.}\label{lambda_{d}ot_{S}}}
\end{eqnarray}

From these expressions, we notice that none of these derivatives depends
on the specific values of the evolving scalar or tensor wavelengths.
From the form of $\gamma$ and $H$ suggested for the IS-bounce in
(\ref{eq:Hubble_IS}) and (\ref{eq:Gamma _IS}) along with the construction
of $A(t)$ and $B(t)$ described in equations (14) and (17) in \cite{Ijjas:2016tpn},
it is possible to compare the evolution of all the physical length
scales $(\ell_{\mathrm{phys}})$ including the sound and particle
horizon: 
\[
{\displaystyle {s_{\mathrm{H}}=\int\frac{c_{s}(t)dt}{a(t)}~;~p_{\mathrm{H}}=\int\frac{dt}{a(t)},}}
\]
through the bouncing phase. In Figure \ref{fig:scalar_lengths}, as
an example, we observe the evolution of the sound horizon and three
sub-horizon physical wavelengths ($\lambda_{\mathrm{phys}}^{\mathcal{S}}$)
corresponding to different comoving modes that propagate through the
so-called acoustic geometry. Such scalar wavelengths never reach the
sound horizon. In the same way, the evolution of the particle horizon
and the corresponding tensor wavelengths ($\lambda_{\mathrm{phys}}^{\mathcal{T}}$)
are depicted in Figure \ref{fig:tensor_lengths}, where each of these
was built from the same comoving modes as in the previous case. In
both scenarios, this indicates that we should not expect any transitions
\textendash{} such as the characteristic ``freezing'' occuring when
a mode crosses the horizon during inflation \textendash{} in the solutions
of the scalar and modes since none of the sub-horizon modes cross
to super-horizon scales. Throughout this paper, we write the physical
wavenumbers in units of the Ricci curvature $|R|^{1/2}$.

After considering these results, we evolve the mode amplitudes for
scalar and tensor fluctuations using the equations of motion in (\ref{eq:IS_amp_S})
and its tensor counterpart in (\ref{eq:IS_amp_T}) considering the
initial conditions in (\ref{eq:IS_real_ICs_scalar}) and (\ref{eq:IS_real_ICs_tensor}).
In Figure \ref{fig:amplitudes}, we represent the dynamics of the
amplitudes of some of the original scalar (in Fig.~\ref{fig:scalar_amps})
and tensor modes (in Fig.~\ref{fig:tensor_amps}) after using the
inversion relationships in (\ref{eq:IS_amplitudes}). This confirms
our previous statement about the lack of a ``mode freezing'' phase
in all the sub-horizon modes. It is possible to observe oscillations
of the modes for the sub-horizon modes with the smallest values of
$k_{\mathrm{phys}}$. In the next subsection of this manuscript, it
will be possible to identify these oscillations in the scalar and
tensor power spectrum and in the occupation numbers at given frequencies.
In the limit where $k$ goes to zero, the dominant scales are the
negative values of $\ddot{z}/z$ and $\ddot{y}/y$ which behave as
time-dependent effective masses.

\subsection{Evolution of the scalar and tensor power spectrum and particle production}

In this subsection, we evaluate the scale dependence and other specific
features of the power spectrum of scalar and tensor perturbations
through the IS bouncing trajectory from the choice of instantaneous
minimal energy initial conditions set in (\ref{eq:IS_real_ICs_scalar})
and (\ref{eq:IS_real_ICs_tensor}) right before the bounce. In addition
to this, we also calculate the evolution of the occupation number
of scalar and tensor fluctuations, observing that some of the features
of the spectra are due to particle production. The power spectra of
primordial scalar and tensor fluctuations follow from the typical
definition given by, 
\begin{equation}
P_{\mathcal{R}}(k)=\frac{k^{3}}{2\pi^{2}}|\mathcal{R}_{\mathbf{k}}(t)|^{2}~~,~~P_{h}(k)=\frac{k^{3}}{2\pi^{2}}\sum_{p=+,\times}|h_{\mathbf{k}}^{p}(t)|^{2},\label{eq:IS_spectrum}
\end{equation}
where the time evolution of the amplitudes $|\mathcal{R}_{\mathbf{k}}(t)|^{2}$
and $|h_{\mathbf{k}}^{p}(t)|^{2}$ follows from the numerical solutions
for $L_{\mathcal{S}}$ and $L_{\mathcal{T}}$ described in (\ref{eq:IS_amp_S})
and (\ref{eq:IS_amp_T}) respectively. The calculations for particle
production become much simpler to evaluate after we rephrase the equations
of motion as in (\ref{eq:IS_eqmov_scalar}) and (\ref{eq:IS_eqmov_tensor}).
We define the occupation number as, 
\begin{equation}
\langle n_{\mathbf{k}}(t)\rangle=\frac{a^{4}A(t)}{2kc_{s}}\left[|\dot{\mathcal{R}}_{\mathbf{k}}|^{2}+\frac{k^{2}c_{s}^{2}}{a^{2}}|\mathcal{R}_{\mathbf{k}}|^{2}\right]-\frac{1}{2},\label{eq:IS_number}
\end{equation}

where $1/2$ corresponds to the minimum possible energy of the system
for the production of scalar fluctuations. The tensor counterpart
is defined in an analog way. In contrast with the definitions in \cite{Barnaby:2010ke},
the last expression is not built from the time-dependent harmonic
oscillators in (\ref{eq:IS_eqmov_scalar}) and (\ref{eq:IS_eqmov_tensor})
in order to not make these quantities divergent in the limit where
$\omega_{\mathcal{S}}^{2}$ and $\omega_{\mathcal{T}}^{2}$ vanish.
In Figure \ref{fig:sp_amps}, we evaluate the dependence of the amplitudes
with $k$ in four different instants of time. We evaluated the scalar
and tensor power spectra in (\ref{eq:IS_spectrum}), showing that
none of the spectra is scale invariant.

We must remark that the amplitude of the tensor modes dominates over
the scalar amplitudes through most of the bouncing phase, as we can
see in figures \ref{fig:sp_0}, \ref{fig:sp_1} and \ref{fig:sp_2}.
This is due to the evolution of the speed of sound, where the enhancement
of the scalar power at later times \textendash{} as shown in Figure
\ref{fig:sp_3} \textendash{} coincides with the final decreasing
phase of the speed of sound at $t\sim50t_{\mathrm{Pl}}$. Scalar and
tensor amplitudes as depicted in these figures show a blue spectrum,
consistent in the high $k$ limit with the results presented in \cite{Creminelli:2010ba}.
The creation of particles in the scalar spectrum at later times is
a visible feature of the scalar curvature spectrum, and it persists
with less intensity as wavelengths become smaller. In addition to
this, it is also possible to observe small oscillations in the tensor
spectrum, which are evidence of particle production for these modes.
We observe the evolution of the occupation number defined in (\ref{eq:IS_number})
in Figure \ref{fig:n_k_t}. 

Here, the production of tensor fluctuations in Figure \ref{fig:tensor_nk}
is much smaller than the oscillations seen in Figure \ref{fig:scalar_nk}
in the same way the oscillations of the tensor amplitudes are smaller
than the corresponding oscillations for the scalar power in Figure
\ref{fig:sp_amps}.

\section{Conclusions}

\acknowledgments It is a pleasure to thank Slava Mukhanov for useful
discussions. The work of D.D. was funded by the Undergraduate Research
grant provided by the Natural Sciences and Engineering Research Council
of Canada. A.F and J.G. want to acknowledge the support of the Discovery
Grants by the Natural Sciences and Engineering Research Council of
Canada, J.G was partially funded by the Billy Jones scholarship granted
by the Department of Physics at Simon Fraser University and by the
Perimeter Institute for Theoretical Physics. Research at the Perimeter
Institute is supported by the Government of Canada through the Department
of Innovation, Science and Economic Development Canada. The work of
A.V. and S.R. was supported by the funds from the European Regional
Development Fund and the Czech Ministry of Education, Youth and Sports
(M\v{S}MT): Project CoGraDS - CZ.02.1.01/0.0/0.0/15\_003/0000437.
A.V. also acknowledges support from the J. E. Purkyn\v{e} Fellowship
of the Czech Academy of Sciences. 

\bibliographystyle{utphys}
\addcontentsline{toc}{section}{\refname}\bibliography{IS_bounce}

\end{document}
