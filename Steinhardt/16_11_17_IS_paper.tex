%% LyX 2.2.3 created this file.  For more info, see http://www.lyx.org/.
%% Do not edit unless you really know what you are doing.
\documentclass[11pt,a4paper,english,american]{article}
\usepackage{lmodern}
\renewcommand{\sfdefault}{lmss}
\renewcommand{\ttdefault}{lmtt}
\usepackage[T1]{fontenc}
\usepackage[latin9]{inputenc}
\usepackage{amsmath}

\makeatletter

%%%%%%%%%%%%%%%%%%%%%%%%%%%%%% LyX specific LaTeX commands.
\special{papersize=\the\paperwidth,\the\paperheight}


%%%%%%%%%%%%%%%%%%%%%%%%%%%%%% Textclass specific LaTeX commands.
\usepackage{jcappub}
\numberwithin{equation}{section}

%%%%%%%%%%%%%%%%%%%%%%%%%%%%%% User specified LaTeX commands.
\pdfoutput=1
\renewcommand\[{\begin{equation}}
\renewcommand\]{\end{equation}} 
\renewcommand*\arraystretch{1.5}

\makeatother

\usepackage{babel}
\begin{document}

\title{Unbraiding the Bounce }

\subheader{preprint number }

\author[a]{David A. Dobre,}

\author[a]{Andrei V. Frolov,}

\author[a,b]{Jos\'{e} T. G\'{a}lvez Ghersi}

\author[c]{Sabir Ramazanov}

\author[c]{and Alexander Vikman}

\affiliation[a]{Department of Physics, Simon Fraser University, }

\affiliation{8888 University Drive, Burnaby, British Columbia V5A 1S6, Canada\\
 }

\affiliation[b]{Perimeter Institute for Theoretical Physics,}

\affiliation{31 Caroline Street North, Waterloo, Ontario, N2L 2Y5, Canada\\
}

\affiliation[c]{\foreignlanguage{english}{CEICO-Central European Institute for Cosmology
and Fundamental Physics, }}

\affiliation{Institute of Physics, the Academy of Sciences of the Czech Republic,
\\
Na Slovance 2, 182 21 Prague 8, Czech Republic\\
}

\emailAdd{ddobre@sfu.ca}

\emailAdd{frolov@sfu.ca}

\emailAdd{joseg@sfu.ca}

\emailAdd{ramazanov@fzu.cz}

\emailAdd{vikman@fzu.cz}

\abstract{We study a recently proposed by Ijjas and Steinhardt particular realization
\cite{Ijjas:2016tpn} of the cosmological bounce scenario. First,
we reveal the exact construction of the Lagrangian used in \cite{Ijjas:2016tpn}.
This explicit construction allowed us to study other cosmological
solutions in this theory. In particular we found solutions with superluminal
speed of sound and discuss the consequences of this feature for a
possible UV-completion. Further, following the originally constructed
background history, we evaluated the tensor and scalar spectra during
the bouncing phase characterized by the violation of null (and strong)
energy condition. We found that the change of the speed of sound is
the cause of the dominance of the tensor power spectrum over the scalar
part through most of the bounce. Moreover, we observe that none of
the spectra evaluated across the bouncing phase is scale invariant.
In addition to this, we present our results for particle production
by showing the evolution of the occupation number of scalar fluctuations
through the bounce. }
\maketitle

\section{Introduction}

\acknowledgments The work of A.V. was supported by the J. E. Purkyn\v{e}
Fellowship of the Czech Academy of Sciences and by the funds from
Project CoGraDS - CZ.02.1.01/0.0/0.0/15\_003/0000437 from the European
Regional Development Fund and the Czech Ministry of Education, Youth
and Sports (M\v{S}MT). 

\bibliographystyle{utphys}
\addcontentsline{toc}{section}{\refname}\bibliography{IS_bounce}

\end{document}
